%        File: !comp!expand("%:p:t")!comp!
%     Created: !comp!strftime("%a %b %d %I:00 %p %Y ").substitute(strftime('%Z'), '\<\(\w\)\(\w*\)\>\(\W\|$\)', '\1', 'g')!comp!
% Last Change: !comp!strftime("%a %b %d %I:00 %p %Y ").substitute(strftime('%Z'), '\<\(\w\)\(\w*\)\>\(\W\|$\)', '\1', 'g')!comp!
%
\documentclass[a4paper]{article}
\usepackage[catalan]{babel}
\usepackage[utf8]{inputenc}
\usepackage[obeyspaces]{url}
\usepackage{comment}
\usepackage[hidelinks]{hyperref}
\usepackage{listings}
\usepackage{xcolor}
\begin{document}
\title{Apunts mitjans de transmissió de dades}
\maketitle

% más info: https://en.wikibooks.org/wiki/LaTeX/Source_Code_Listings
\lstset{%language=bash,
	backgroundcolor=\color{white},
	basicstyle=\footnotesize,
	breaklines=true, 
	commentstyle=\color{brown},
	% si queremos marco en el código
	frame=single,
}
 
% ----------- ejemplos de código fuente
% codigo en la misma linea carmela \lstinline! sudo apt-get install apache2!
%
% \begin{lstlisting}
% # ejecuto en la linea de comandos como sudoer
% sudo apt-get install apache2 
% \end{lstlisting}

\begin{comment}
oddsidemargin \the\oddsidemargin \newline
textwidth \the\textwidth \newline
marginparsep \the\marginparsep \newline
marginparwidth \the\marginparwidth \newline
hoffset \the\hoffset \newline
paperwidth \the\paperwidth 
\end{comment}

\section{Ones}
Les ones són un tipus de moviment que no desplacen partícules però sí propietats de l'espai, per exemple, densitat, pressió o camp magnètic. Hi han dos tipus d'ones, les ones transversals i les ones longitudinals, i es diferencien en com es comporten les vibracions d'aquestes ones. En les ones transversals les vibracions són perpendiculars a la direcció de propagació de l'ona i en les longitudinals són paralel·les a la direcció de propagació de l'ona.

Exemples d'ones transversals són les ones wifi i exemple d'ones longitudinals \'{e}s el moviment que es crea en deixar anar una \textit{muelle}.

\section{Cables UTP}
\subsection{Ethernet. Tecnologies i categories}

\begin{tabular}[]{|p{2cm}|p{2cm}|p{2cm}|p{2cm}|p{2cm}|}
	\hline
	\textbf{Tecnologia}&\textbf{Velocitat transmissió}&\textbf{Tipus de cable}&\textbf{Dist. màxima}&\textbf{Tecnologia} \\
	\hline\hline
	10Base2	&10 Mbit/s	&Coaxial	&185 m	&Bus (Conector T) \\
	\hline
	10BaseT	&10 Mbit/s	&Par Trenzado	&100 m	&Estrella (Hub o Switch)\\
	\hline
	10BaseF	& 10 Mbit/s	& Fibra óptica	& 2000 m	& Estrella (Hub o Switch) \\
	\hline
	100BaseT4	& 100 Mbit/s	& Par Trenzado (categoría 3UTP)	& 100 m	& Estrella. Half Duplex (hub) y Full Duplex (switch) \\
	\hline
	100BaseTX	& 100 Mbit/s	& Par Trenzado (categoría 5UTP)	& 100 m	& Estrella. Half Duplex (hub) y Full Duplex (switch) \\
	\hline
	100BaseFX	& 100 Mbit/s	& Fibra óptica	& 2000 m	& No permite el uso de hubs \\
	\hline
	1000BaseT	& 1000 Mbit/s	& (categoría 5e ó 6UTP )	& 100 m	& Estrella. Full Duplex (switch) \\
	\hline
	1000BaseSX	& 1000 Mbit/s	& Fibra óptica (multimodo)	& 550 m	& Estrella. Full Duplex (switch) \\
	\hline
	1000BaseLX	& 1000 Mbit/s	& Fibra óptica (monomodo)	& 5000 m	& Estrella. Full Duplex (switch) \\
	\hline
\end{tabular}

\section{Sobre aquest document}
\textbf{Copyright}\copyright\ \textbf{\the\year\ Juan Aguilera}.\\
Permission is granted to copy, distribute and/or modify this document under the terms of the GNU Free Documentation License, Version 1.3 or any later version published by the Free Software Foundation;\\
with no Invariant Sections, no Front-Cover Texts, and no Back-Cover Texts.\\
A copy of the license is included in the section entitled \href{http://www.gnu.org/licenses/fdl.html}{``GNU Free Documentation License``}.

\begin{thebibliography}{99}
	\bibitem{<++>} 
		% Exemple url: \bibitem{Debian} \url{https://wiki.debian.org/es}%
\end{thebibliography}

\end{document}


