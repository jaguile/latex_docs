%        File: !comp!expand("%:p:t")!comp!
%     Created: !comp!strftime("%a %b %d %I:00 %p %Y ").substitute(strftime('%Z'), '\<\(\w\)\(\w*\)\>\(\W\|$\)', '\1', 'g')!comp!
% Last Change: !comp!strftime("%a %b %d %I:00 %p %Y ").substitute(strftime('%Z'), '\<\(\w\)\(\w*\)\>\(\W\|$\)', '\1', 'g')!comp!
%
\documentclass[a4paper]{article}
\usepackage[catalan]{babel}
\usepackage[utf8]{inputenc}
\usepackage[obeyspaces]{url}
\usepackage{comment}
\usepackage[hidelinks]{hyperref}
\usepackage{listings}
\usepackage{xcolor}
\begin{document}
\title{Teoria de la comunicació}
\maketitle

% más info: https://en.wikibooks.org/wiki/LaTeX/Source_Code_Listings
\lstset{%language=bash,
	backgroundcolor=\color{white},
	basicstyle=\footnotesize,
	breaklines=true, 
	commentstyle=\color{brown},
	% si queremos marco en el código
	frame=single,
}

% ----------- ejemplos de código fuente
% codigo en la misma linea carmela \lstinline! sudo apt-get install apache2!
%
% \begin{lstlisting}
% # ejecuto en la linea de comandos como sudoer
% sudo apt-get install apache2 
% \end{lstlisting}

\begin{comment}
oddsidemargin \the\oddsidemargin \newline
textwidth \the\textwidth \newline
marginparsep \the\marginparsep \newline
marginparwidth \the\marginparwidth \newline
hoffset \the\hoffset \newline
paperwidth \the\paperwidth 
\end{comment}

\section{Comunicació serial}<++>
\section{Frame relay}<++>
\section{Circuit Virtual}<++>
\section{DTE & DCE}<++>
\section{Conmutació de paquets}<++>
\section{Fiber Distributed Data Interface}
\section{Fiber channel over ethernet}<++>


\section{Sobre aquest document}
\textbf{Copyright}\copyright\ \textbf{\the\year\ Juan Aguilera}.\\
Permission is granted to copy, distribute and/or modify this document under the terms of the GNU Free Documentation License, Version 1.3 or any later version published by the Free Software Foundation;\\
with no Invariant Sections, no Front-Cover Texts, and no Back-Cover Texts.\\
A copy of the license is included in the section entitled \href{http://www.gnu.org/licenses/fdl.html}{``GNU Free Documentation License``}.

\begin{thebibliography}{99}
	\bibitem{<++>} 
		% Exemple url: \bibitem{Debian} \url{https://wiki.debian.org/es}%
\end{thebibliography}

\end{document}


