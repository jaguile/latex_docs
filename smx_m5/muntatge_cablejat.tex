%        File: !comp!expand("%:p:t")!comp!
%     Created: !comp!strftime("%a %b %d %I:00 %p %Y ").substitute(strftime('%Z'), '\<\(\w\)\(\w*\)\>\(\W\|$\)', '\1', 'g')!comp!
% Last Change: !comp!strftime("%a %b %d %I:00 %p %Y ").substitute(strftime('%Z'), '\<\(\w\)\(\w*\)\>\(\W\|$\)', '\1', 'g')!comp!
%
\documentclass[a4paper]{article}
\usepackage[catalan]{babel}
\usepackage[utf8]{inputenc}
\usepackage[obeyspaces]{url}
\usepackage{comment}
\usepackage[hidelinks]{hyperref}
\usepackage{listings}
\usepackage{xcolor}
\begin{document}
\title{<++>}
\maketitle

% más info: https://en.wikibooks.org/wiki/LaTeX/Source_Code_Listings
\lstset{%language=bash,
	backgroundcolor=\color{white},
	basicstyle=\footnotesize,
	breaklines=true, 
	commentstyle=\color{brown},
	% si queremos marco en el código
	frame=single,
}

% ----------- ejemplos de código fuente
% codigo en la misma linea carmela \lstinline! sudo apt-get install apache2!
%
% \begin{lstlisting}
% # ejecuto en la linea de comandos como sudoer
% sudo apt-get install apache2 
% \end{lstlisting}

\begin{comment}
oddsidemargin \the\oddsidemargin \newline
textwidth \the\textwidth \newline
marginparsep \the\marginparsep \newline
marginparwidth \the\marginparwidth \newline
hoffset \the\hoffset \newline
paperwidth \the\paperwidth 
\end{comment}

\section{Materials}

\begin{itemize}
	\item Tipus de cable
	\item Canaletes, angles, derivadors, etc.
	\item \textit{pela cables}
	\item Crimpador
	\item Taladre
	\item tornavís
	\item cargols, arandel·les
	\item aparell per ficar els cables en la roseta famella
	\item \textit{Càncamos} per poder posar els cargols
\end{itemize}<++>

\section{Canaletes}
Tinguem en compte:

\begin{itemize}
	\item Canal
	\item Angles
	\item Derivadors
	\item Canal estret d'un cable
	\item Canal estret amb separació
\end{itemize}<++>

\section{Procès}

\subsection{Planificació}

\begin{itemize}
	\item Mesurar el que ens cal per cada paret
	\item angles
\end{itemize}<++>

\subsection{Fer els forats a la paret}

Els forats es fan al llarg de tota la línia per on han de passar les canaletes. Un forat a cada metre o metre i mig just a la meitat de l'amplada de la canaleta.

\subsection{Serrar les canaletes}<++>

\subsection{Fixar les canaletes en la paret}

\begin{itemize}
	\item Pegant-les
	\item fent forats
\end{itemize}

\subsection{Com fer forats amb el taladre}

\begin{itemize}
	\item Broca a utilitzar
	\item Seguretat (guants, ulleres, etc)
\end{itemize}<++>

\subsection{grimpar}<++>

\subsection{Fer arribar tots els cables al rack}<++>

\section{Preguntes}

\begin{enumerate}
	\item Com han de ser de grans les canaletes? Potser s'ha de calcular segons els cables que han de passar per ella? Un cable per port?
	\item Una canaleta per grup?
	\item Com s'ha de fer el prototip, per a que passi nom\'{e}s un cable?
\end{enumerate}<++>

\section{Sobre aquest document}
\textbf{Copyright}\copyright\ \textbf{\the\year\ Juan Aguilera}.\\
Permission is granted to copy, distribute and/or modify this document under the terms of the GNU Free Documentation License, Version 1.3 or any later version published by the Free Software Foundation;\\
with no Invariant Sections, no Front-Cover Texts, and no Back-Cover Texts.\\
A copy of the license is included in the section entitled \href{http://www.gnu.org/licenses/fdl.html}{``GNU Free Documentation License``}.
\begin{thebibliography}{99}
	\bibitem{<++>} 
		% Exemple url: \bibitem{Debian} \url{https://wiki.debian.org/es}%
		\bibitem{canaletes}\url{http://3didshop.com/canaletas-bloque0.html}
		\bibitem{Muntatge}\url{https://www.youtube.com/watch?v=zqGIhLY8Sns}


\end{thebibliography}

\end{document}


