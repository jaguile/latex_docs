%        File: ssh.tex
%     Created: mié mar 02 11:00  2016 C
% Last Change: mié mar 02 11:00  2016 C
%
\documentclass[a4paper]{article}
\usepackage[catalan]{babel}
\usepackage[utf8]{inputenc}
\usepackage[obeyspaces]{url}
\usepackage{comment}
\begin{document}
\title{SSH i eines d'acc\'es remot}
\author{Juan Aguilera}
\maketitle

% \begin{comment}
% oddsidemargin \the\oddsidemargin \newline
% textwidth \the\textwidth \newline
% marginparsep \the\marginparsep \newline
% marginparwidth \the\marginparwidth \newline
% hoffset \the\hoffset \newline
% paperwidth \the\paperwidth 
% \end{comment}

% TODO APUNTS DRIVE SOBRE SSH I ACCÉS REMOT
% TODO DIsseny d'apunts ssh de la web en latex
% TODO PROVAR GNU Screen
% TODO PROVAR LES EINES GRÀFIQUES D'ACCÉS REMOT
% TODO NAGIOS | ANSILE | PUPPET | OCS INVENTORY - YAML / JSON

\section{GNU Screen}
\subsection{Què podem fer amb GNU Screen}

\begin{enumerate}
	\item Executar múltiples consoles a la vegada
	\item associar/desassociar una consola remota
	\item assistència remota per terminal
\end{enumerate}

\subsection{Conceptes}

\textbf{Sessió} \'es l'àrea de treball que es crea quan executes la comanda \verb+screen+.

\textbf{Finestra} \'es una consola oberta dintre d'una sessió. Poden haver-hi múltiples.

\textbf{Regió} \'es una divisió d'una finestra. Dintre de cada regio pot haver-hi una altra finestra. \textbf{Split} \'es la porció de la regió.

% http://www.linux.com/learn/tutorials/442418-using-screen-for-remote-interaction
% http://gnuscreen.org/
% http://www.gnu.org/software/screen/manual/html_node/index.html#Top

\section{Proves amb GNU Screen}

\begin{enumerate}
	\item Assistència remota per consola: obres sessió screen; activas el mode multiusuari; l'altre es connecta des de una altra màquina per ssh, per exemple (o amb un altre usuari); des de la màquina remota fas screen -x <el nom que li has donat a la sessió>
	\item Idem però ara l'usuari que es connecta \ès diferent usuari local. No me ha funcionado. Hay que dar permiso a la otra persona y previamente dar el setuid al binario.
\end{enumerate}

\begin{thebibliography}{99}
	\bibitem{WIKI_NX} \urL{https://en.wikipedia.org/wiki/NX_technology}
	\bibitem{WIKI_RESUM} \url{https://en.wikipedia.org/wiki/Comparison_of_remote_desktop_software}
	\bibitem{WIKI_GNUSCN} \url{https://en.wikipedia.org/wiki/GNU_Screen} 
	\bibitem[<+biblabel+>]{GNUSCN} \url{https://www.gnu.org/software/screen/}
	\bibitem{WIKI_ES_GNUSCN} \url{https://es.wikipedia.org/wiki/GNU_Screen}
	\bibitem[]{BLOG_GNUSCN} \url{https://phenobarbital.wordpress.com/2013/02/18/linux-usando-gnu-screen/}
		% Exemple url: \bibitem{Debian} \url{https://wiki.debian.org/es}%
\end{thebibliography}

\end{document}


