\documentclass[a4paper]{article}
\usepackage[catalan]{babel}
\usepackage[utf8]{inputenc}
\usepackage[obeyspaces]{url}
\usepackage{comment}
\begin{document}
\title{Systemd}
\author{Juan Aguilera}
\maketitle

\begin{comment}
oddsidemargin \the\oddsidemargin \newline
textwidth \the\textwidth \newline
marginparsep \the\marginparsep \newline
marginparwidth \the\marginparwidth \newline
hoffset \the\hoffset \newline
paperwidth \the\paperwidth 
\end{comment}

\section{Comanda bàsica}
systemctl és la comanda per gestionar totes les unitats amb systemd. Exemples:
\begin{itemize}
	\item Per veure l'estat de la unitat  \newline
	\verb+# systemctl start/stop/reload/status <unidad>+ 
	\item activar l'inici automàtic en l'arrancada del sistema. \newline
	\verb+# systemctl enable unidad+ 
\end{itemize}

\section{Unitats}

Les unitats poden ser serveis, punts de muntatge, dispositius, sockets. Els tipus d'unitats es denoten amb el sufixe: <nom de la unitat>.tipus, on tipus pot ser service, mount, socket o device.

\begin{thebibliography}{99}
	\bibitem{Arch} \url{https://wiki.archlinux.org/index.php/Systemd_%28Espa%C3%B1ol%29}
	\bibitem{Debian} \url{https://wiki.debian.org/es/systemd}
	\bibitem{PROCESS_CONTROL} \url{https://www.linux.com/learn/cleaning-your-linux-startup-process}
	\bibitem{SYSTEMD_INTRO} \url{https://www.linux.com/learn/here-we-go-again-another-linux-init-intro-systemd}
	\bibitem{SYSTEMD_INTRO} \url{https://www.linux.com/learn/here-we-go-again-another-linux-init-intro-systemd}
	\bibitem{SYSTEMD_COMMANDS} \url{https://www.linux.com/learn/intro-systemd-runlevels-and-service-management-commands}
	\bibitem{SYSTEMD_UNDERSTANDING} \url{https://www.linux.com/learn/understanding-and-using-systemd}
	\bibitem{Monitoring}\url{https://www.linux.com/learn/linux-system-monitoring-and-more-auditd}
		

	
\end{thebibliography}
\end{document}
