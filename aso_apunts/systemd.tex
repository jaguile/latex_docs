\documentclass[a4paper]{article}
\usepackage[catalan]{babel}
\usepackage[utf8]{inputenc}
\begin{document}
\title{Systemd}
\author{Juan Aguilera}
\maketitle

\section{Referències}
En quant es pugui, passar-ho a bibliografia latex.
\begin{verbatim}
https://wiki.archlinux.org/index.php/Systemd_%28Espa%C3%B1ol%29
https://wiki.debian.org/es/systemd
\end{verbatim}
\section{comanda bàsica}
systemctl és la comanda per gestionar totes les unitats amb systemd.

\begin{verbatim}
# systemctl start/stop/reload/status unidad
# systemctl enable unidad --> activar l'inici automàtic en l'arrancada del sistema.
\end{verbatim}
\section{Unitats}

Les unitats poden ser serveis, punts de muntatge, dispositius, sockets. Els tipus d'unitats es denoten amb el sufixe: <nom de la unitat>.tipus, on tipus pot ser service, mount, socket o device.

\end{document}
