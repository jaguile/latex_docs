%        File: !comp!expand("%:p:t")!comp!
%     Created: !comp!strftime("%a %b %d %I:00 %p %Y ").substitute(strftime('%Z'), '\<\(\w\)\(\w*\)\>\(\W\|$\)', '\1', 'g')!comp!
% Last Change: !comp!strftime("%a %b %d %I:00 %p %Y ").substitute(strftime('%Z'), '\<\(\w\)\(\w*\)\>\(\W\|$\)', '\1', 'g')!comp!
%
\documentclass{beamer}
 
\usepackage[utf8]{inputenc}
\usepackage[catalan]{babel}
\usepackage[obeyspaces]{url}
\usepackage{comment}
\usepackage{hyperref}
\usepackage{lmodern}
\usepackage[T1]{fontenc}
\usetheme{Madrid}
\usecolortheme{seahorse}
 
%------------------------------------------------------------
%Information to be included in the title page:
\title[<++>]{<++>}
\subtitle{<++>}
\author{<++>}
\institute[<++>]{<++>}
\date{<++>}
\logo{\includegraphics[height=1cm]{Logo-Consorci-JdA.eps}}
%------------------------------------------------------------
 
%------------------------------------------------------------
% slide en cada secció amb la llista i subratyat la secció que toca
\AtBeginSection[]
{
	\begin{frame}
		\frametitle{Table of Contents}
		\tableofcontents[currentsection]
	\end{frame}
}
%------------------------------------------------------------

\begin{document}

% amb aquesta línia inserto un primer slide amb el títol, autor, etc
% 
\frame{\titlepage}

%------------------------------------------------------------
% Inserta slide amb la llista de continguts
\begin{frame}
	\frametitle{Índex de continguts}
	\tableofcontents
\end{frame}
%------------------------------------------------------------

\section{<++>}

%------------------------------------------------------------
% OJO: frame no es igual a slide. Un frame pot contenir mes d'un slide
\begin{frame}
	\frametitle{<++>}
	% Aqui va el contingut de la diapo
	% cada vegade que vulguis crear una transició, fica 
	% \pause
	%
\end{frame}
%------------------------------------------------------------
 
%------------------------------------------------------------
% Exemple de blocs d'exemple, remarks, i paraules i blocs 
% importants
%
% \begin{frame}
% 	\frametitle{Diapos 3 de la segona secció}
% 	Anem a veure el bàsic \pause per crear una presentació \pause amb Beamer
% 	Hi ara anem a marcar alguna \alert{cosa important}
% 
% 	\begin{block}{Remark}
% 		Sample text
% 	\end{block}
% 	 
% 	\begin{alertblock}{Important theorem}
% 		Sample text in red box
% 	\end{alertblock}
% 	 
% 	\begin{examples}
% 		Sample text in green box. ``Examples'' is fixed as block title.
% 	\end{examples}
% 
% \end{frame}
%------------------------------------------------------------

%------------------------------------------------------------
% Exemple de diapo amb dues columnes
% \begin{frame}
% 	\frametitle{Diapos 4 de la segona secció}
% 	\begin{columns}
% 		\begin{column}{5cm}
% 		Això \'es una primera columna. \\
% 		waaahhh
% 	\end{column}
% 	\begin{column}{5cm}
% 		Això \'es una segona columna. \\
% 		waaahhh
% 	\end{column}
% 	\end{columns}
% \end{frame}
%------------------------------------------------------------

%------------------------------------------------------------
% Exemple de diapo amb text verbatim i de text al peu
% \begin{frame}
%   \frametitle{Diapos 5 de la segona secció}
% 	Exemple de com ficar text en verbatim\footnote{On a fast machine.}
% 	\begin{semiverbatim}
% 		\$ sudo apt-get install apache2
% 	\end{semiverbatim}
% \end{frame}
%------------------------------------------------------------

%------------------------------------------------------------
\begin{frame}
	\frametitle{Referències}
	\begin{thebibliography}{99}
			% \bibitem{sharelatex} \url{https://www.sharelatex.com/learn/Beamer}
		\bibitem{<++>} \url{<++>}
	\end{thebibliography}
\end{frame}
%------------------------------------------------------------

%------------------------------------------------------------
\begin{frame}
	\frametitle{Sobre aquest document}
	\textbf{Copyright}\copyright\ \textbf{\the\year\ Juan Aguilera}.\\
	Permission is granted to copy, distribute and/or modify this document under the terms of the GNU Free Documentation License, Version 1.3 or any later version published by the Free Software Foundation;\\
	with no Invariant Sections, no Front-Cover Texts, and no Back-Cover Texts.\\
	A copy of the license is included in the section entitled \href{http://www.gnu.org/licenses/fdl.html}{``GNU Free Documentation License``}.
\end{frame}
%------------------------------------------------------------

\end{document}


