%        File: mySecondBeamerPresentation.tex
%     Created: dom may 29 08:00  2016 C
% Last Change: dom may 29 08:00  2016 C
%
\documentclass{beamer}
 
\usepackage[utf8]{inputenc}
\usepackage[catalan]{babel}
\usepackage[obeyspaces]{url}
\usepackage{comment}
\usepackage{hyperref}
\usepackage{lmodern}
\usepackage[T1]{fontenc}
\usetheme{Madrid}
\usecolortheme{seahorse}
 
%Information to be included in the title page:
\title[Beamer apunts]{Presentacions amb Latex}
\subtitle{Beamer class}
\author{Juan Aguilera}
\institute[ins jda]{Institut Joan d'Àustria}
\date{2016}
\logo{\includegraphics[height=1cm]{Logo-Consorci-JdA.eps}}
 
% slide en cada secció amb la llista i subratyat la secció que toca
\AtBeginSection[]
{
  \begin{frame}
	    \frametitle{Table of Contents}
	    \tableofcontents[currentsection]
  \end{frame}
}

\begin{document}

% amb aquesta línia inserto un primer slide amb el títol, autor, etc
% 
\frame{\titlepage}

% Inserta slide amb la llista de continguts
\begin{frame}
 	\frametitle{Índex de continguts}
 	\tableofcontents
\end{frame}

\section{Primera secció}

% OJO: frame no es igual a slide. Un frame pot contenir mes d'un slide
\begin{frame}
	\frametitle{La meva primera diapo amb Beamer}
	Anem a veure el bàsic per crear una presentació amb Beamer
\end{frame}
 
\section{Secció dos}

\begin{frame}
	\frametitle{Diapos de la segona secció}
	Una altra diapo amb un altre títol
\end{frame}
 
\begin{frame}
	\frametitle{Diapos 2 de la segona secció}
	Anem a veure el bàsic \pause per crear una presentació \pause amb Beamer
\end{frame}

\begin{frame}
	\frametitle{Diapos 3 de la segona secció}
	Anem a veure el bàsic \pause per crear una presentació \pause amb Beamer
	Hi ara anem a marcar alguna \alert{cosa important}

	\begin{block}{Remark}
		Sample text
	\end{block}
	 
	\begin{alertblock}{Important theorem}
		Sample text in red box
	\end{alertblock}
	 
	\begin{examples}
		Sample text in green box. ``Examples'' is fixed as block title.
	\end{examples}

\end{frame}

\begin{frame}
	\frametitle{Diapos 4 de la segona secció}
	\begin{columns}
		\begin{column}{5cm}
		Això \'es una primera columna. \\
		waaahhh
	\end{column}
	\begin{column}{5cm}
		Això \'es una segona columna. \\
		waaahhh
	\end{column}
	\end{columns}
\end{frame}

\begin{frame}
	\frametitle{Diapos 5 de la segona secció}
	Exemple de com ficar text en verbatim\footnote{On a fast machine.}
	\begin{semiverbatim}
		\$ sudo apt-get install apache2
	\end{semiverbatim}
\end{frame}

% \section{Referències}
\begin{frame}
	\frametitle{Referències}
	\begin{thebibliography}{99}
		\bibitem{sharelatex} \url{https://www.sharelatex.com/learn/Beamer}
	\end{thebibliography}
\end{frame}
%https://www.sharelatex.com/learn/Beamer
%http://tex.stackexchange.com/questions/17734/cannot-determine-size-of-graphic
\end{document}
