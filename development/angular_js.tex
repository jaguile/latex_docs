%        File: !comp!expand("%:p:t")!comp!
%     Created: !comp!strftime("%a %b %d %I:00 %p %Y ").substitute(strftime('%Z'), '\<\(\w\)\(\w*\)\>\(\W\|$\)', '\1', 'g')!comp!
% Last Change: !comp!strftime("%a %b %d %I:00 %p %Y ").substitute(strftime('%Z'), '\<\(\w\)\(\w*\)\>\(\W\|$\)', '\1', 'g')!comp!
%
\documentclass[a4paper]{article}
\usepackage[catalan]{babel}
\usepackage[utf8]{inputenc}
\usepackage[obeyspaces]{url}
\usepackage{comment}
\usepackage{hyperref}
\begin{document}
\title{Angular.js}
\maketitle

\begin{comment}
oddsidemargin \the\oddsidemargin \newline
textwidth \the\textwidth \newline
marginparsep \the\marginparsep \newline
marginparwidth \the\marginparwidth \newline
hoffset \the\hoffset \newline
paperwidth \the\paperwidth 
\end{comment}

\section{Desenvolupament d'aplicacions mòbils hídrides amb IONIC}

 	Ionic s un framework d’aplicacions mòbils HTML5 dirigit a la creació d’aplicacions mòbils híbrides. Convertint-se en un dels frameworks del moment per utilitzar AngularJS per gestionar les aplicacions assegurant apps ràpides i escalables, s considerat per molts com el “Bootstrap per mòbils”. 

	Aquest curs està adreçat a professorat que desitgi adquirir coneixements i habilitats en el desenvolupament i creació d’aplicacions mòbils híbrides mitjançant Ionic amb tecnologies que ja són conegudes, com HTML, CSS i Javascript.

	Requisits: Per un correcte aprofitament del curs es requereix que es disposin de coneixements previs en HTML i Javascript. Tamb es recomanen coneixements en AngularJS tot i que no s obligatori. 

Mòdul 1: Introducció a Ionic i les aplicacions híbrides
Mòdul 2: Configuració de l’entorn de desenvolupament
Mòdul 3: Coneixements necessaris d’Angular JS
Mòdul 4: Navegació amb Ionic i components core
Mòdul 5: Tabs, llistes avançades i components de formularis
Mòdul 6: Aplicació del Temps, utilitzant menús laterals, modals, action sheets i ionScroll
Mòdul 7: Tècniques avançades per aplicacions professionals
Mòdul 8: Ús de plugins de Cordova
Mòdul 9: Previsió, depuració i automatització del testing
Mòdul 10: Construir i publicar aplicacions
\section{Angular.js i node.js}
AngularJS, s un framework de codi obert, concebut i dissenyat per abordar molts dels problemes trobats en el desenvolupament de single-page Applications. El seu objectiu s simplificar el desenvolupament i les proves d’aquest tipus d’aplicacions, proporcionant un marc pel costat del client en arquitectures model-vista-controlador (MVC), juntament amb els components ms utilitzats en aplicacions d’Internet.

Aquest curs parteix d’un coneixement inicial sobre tecnologia web sòlid i acompanya l’alumne des de l’inici fins crear una aplicació completa utilitzant les característiques bàsiques i avançades d’AngularJS. 

En la segona part del curs s’abordarà la plataforma Node.js, dissenyada pel desenvolupament ràpid d’aplicacions en xarxa escalables. 
El present curs t per objectiu presentar als assistents els coneixements sobre el model I/O basat en esdeveniments pel desenvolupament d’aplicacions de tractament de dades en temps real que funcionen sobre dispositius distribuïts. 
S’exploren les opcions de disseny que fan Node.js únic, canviant la manera de construir aplicacions y com els sistemes d’aplicacions funcionen de manera ms efectiva en aquest model. 
 	 
Continguts  	Mòdul 1: Introducció a la programació d’aplicacions amb AngularJS
1. Presentació
2. Arquitectura de Single Page Applications
3. Breu introducció a git i a git­flow
4. Creació de projectes mitjançant Yeoman
5. Workflows de desenvolupament amb Grunt/ Glup
6. Mòduls AngularJS
7. Data binding entre html i javascript: %$scope%
8. Controladors
9. Injecció de dependències
10. Tècniques de depuració
11. Directives i filtres
12. Twitter Bootstrap i AngularJS
13. Personalització de temes de Bootstrap
14. Internacionalització amb el mòdul translate
15. Promises
16. Consum de serveis Rest amb $http i $resource
17. Interceptors
18. Autentificació d’usuaris
Mòdul 2: Introducció a la programació d’aplicacions amb Node.js
1. Introducció a Node.js
2. Instal•lació de Node.js
3. Modules i NPM
4. El patró Callback 
5. Emissor d’esdeveniments 
6. Gestió d’errors 
7. Buffers
8. Streams
9. HTTP & HTTPS
10. Peticions HTTP 
\section{Sobre aquest document}
\textbf{Copyright}\copyright\ \textbf{\the\year\ Juan Aguilera}.\\
Permission is granted to copy, distribute and/or modify this document under the terms of the GNU Free Documentation License, Version 1.3 or any later version published by the Free Software Foundation;\\
with no Invariant Sections, no Front-Cover Texts, and no Back-Cover Texts.\\
A copy of the license is included in the section entitled \href{http://www.gnu.org/licenses/fdl.html}{``GNU Free Documentation License``}.

\begin{thebibliography}{99}
	\bibitem{<++>} 
		% Exemple url: \bibitem{Debian} \url{https://wiki.debian.org/es}%
\end{thebibliography}

\end{document}


